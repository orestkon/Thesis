%
%% Abstract greek
%
\thispagestyle{empty}
\chapter*{Σύνοψη}
Το κρυπτονόμισμα \hyperref[sec:Bitcoin]{Bitcoin} αποτελεί την πρώτη πετυχημένη εφαρμογή της ιδέας του ηλεκτρονικού χρήματος χωρίς την διαμεσολάβηση τρίτων. Στην πορεία, πολλά κρυπτο- νομίσματα βασίστηκαν στην συγκεκριμένη τεχνολογία, εστιάζοντας το καθένα στους δικούς του στόχους και σκοπούς. Το κρυπτονόμισμα \hyperref[sec:Monero]{Monero} είναι ένα τέτοιο εγχείρημα, βασικός σκοπός του οποίου είναι η διασφάλιση της ιδιωτικότητας και της ανωνυμίας.

Σε έναν κόσμο όπου η παρακολούθηση εντείνεται, το εγχείρημα του Monero σημαί- νει τον συναγερμό για την διαρκή καταπάτηση ενός εκ των θεμελιωδών ανθρώπινων δικαιωμάτων. Επιπλέον, καθώς οι επιχειρήσεις έχουν περιορίσει δραματικά τον υγιή ανταγωνισμό σχεδόν σε όλα τα διαδεδομένα κρυπτονομίσματα, το Monero προσπαθεί να τον διατηρήσει στην κοινότητά του. Ένα από τα δομικά στοιχεία του Monero είναι η διατήρηση της ισότητας μεταξύ των "ανθρακωρύχων" (\hyperref[sec:mining]{miners}), η οποία επιτυγχάνεται μέσω της ισονομίας (\hyperref[sec:egalitarian]{egalitarianism}).

Η ισονομία είναι συνέπεια μιας ιδιότητας της κρυπτογραφικής συνάρτησης που χρησιμοποιείται για την "εξόρυξη" νομισμάτων. Η συνάρτηση που χρησιμοποιείται στο Monero για αυτόν τον σκοπό λέγεται \hyperref[ch:cryptonight]{CryptoNight} και είναι μέρος του \hyperref[sec:CryptoNote]{CryptoNote} πρωτοκόλλου. Το στοιχείο της συνάρτησης που επιτυγχάνει την ισονομία είναι μια κρυπτογραφική ιδιότητα, η οποία ονομάζεται \hyperref[sec:memory-hard]{memory-hardness}. Η CryptoNight συνάρ- τηση θεωρείται ότι διαθέτει αυτήν την ιδιότητα. Όμως, μέχρι σήμερα αυτό παραμένει ισχυρισμός.

Θέλοντας να ελέγξουμε την ορθότητα αυτού του ισχυρισμού, προσπαθήσαμε να κατασκευάσουμε μια μαθηματική απόδειξη. Αναφέρουμε τους λόγους για τους οποίους αποτυγχάνουμε να διατυπώσουμε μία τέτοια απόδειξη και προσπαθούμε να τους χρησι- μοποιήσουμε για να καταρρίψουμε αυτόν τον ισχυρισμό. Απ' όσο γνωρίζουμε, η παρού- σα εργασία είναι η πρώτη που μελετά αυτήν την ιδιότητα για την συνάρτηση CryptoNight και παρουσιάζεται για πρώτη φορά γραφικά η εσωτερική δομή της.

Τέλος, παρουσιάζουμε την γνώση που αποκτήσαμε και ελπίζουμε αυτή η εργασία να φανεί χρήσιμη μελλοντικά σε συναδέλφους που θέλουν να συμβάλλουν στην έρευνα στο ευρύτερο πεδίο. Στόχος αυτής της έρευνας είναι να συνεισφέρει στην προσπάθεια του εγχειρήματος Monero για την διασφάλιση της ιδιωτικότητας, της ανωνυμίας και της ισότητας.
\clearpage
%
%% Empty page
%
\thispagestyle{empty}
\null
\clearpage
