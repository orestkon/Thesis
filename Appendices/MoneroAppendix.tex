\chapter{Monero}
%
\section{CryptoNote elliptic curve} \label{sec:ed25519}
The elliptic curve \emph{ed25519} is both a signature scheme and a use case for Edwards form Curve25519~\cite{eddsa}. EdDSA (\emph{Edwards-curve Digital Signature Algorithm}) generalises this signature scheme to any curve in Edwards form.

\emph{Curve25519} first arrived in 2006~\cite{curve25519}, a few years before the Edwards normal form papers on elliptic curves. Montgomery curves, the form of curve used for Curve25519, was originally used to speed up elliptic curve factorisation~\cite{montgomery}. The original proposal for Curve25519 was for use as a Diffie-Hellman (key exchange) protocol~\cite{Diffie:2006:NDC:2263321.2269104}. This is still its use and is now often called \emph{X25519}.

Later, Edwards came up with his own form of elliptic curve~\cite{edward}. Daniel J. Bernstein, Tanja Lange et al. researched these forms and realised they too were fast, especially for signatures and we got Ed25519~\cite{eddsa} using the Edwards form of Curve25519.

So far we have the following nomenclature:
\begin{description}
  \item [Curve25519, Curve41417, Ed448-Goldilocks] generally the name of the curve itself.
  \item [X25519, X448, X41417] Diffie-Hellman key exchange schemes using the above curve.
  \item [EdDSA, Ed25519, Ed448] The first being the generic Edwards variant of DSA, plus other fixes, the others being specific instances matched to their curve names.
\end{description}

Confusingly, Open Whisper Systems came up with \emph{XEdDSA}~\cite{signal}. To quote them:
\begin{verbatim}
  XEdDSA enables use of a single key pair format for both elliptic
  curve Diffie-Hellman and signatures.
\end{verbatim}

Hence, \emph{X EdDSA} is taken to mean "exchange and EdDSA" of the given curve. In this instance, the key exchange part still happens using the montgomery form of the curve, but the signature part (EdDSA) uses the same curve in Edwards form.

\section{Stealth address construction} \label{sec:construction}
Here is a functional example for deriving a Monero stealth address. Here we will examine the developer mechanics not cryptographic theory. Results below duplicate functionality that is part of \emph{Crypto Note Test Address}~\cite{teststealth}.

It is worth noting custom \verb|bytes_to_words|, \verb|sc_reduce32|, and \verb|secret_key_to_| \verb|public_key| executables (coded in C or C++) below were named after Monero's functions that yielded output results. C++ coding insights came from \verb|main.cpp|. The \verb|bx| command line is bitcoin-explorer, see~\cite{bx}.

Monero's \verb|secret_key_to_public_key()| functionality is using \emph{Ed25519} technology but not in a inclusive way. Only the necessary computations for the production of stealth addresses are implemented. Results are different from \emph{TOR} test vectors results that custom executables utilizing \emph{libsodium} and \emph{ed25519-donna} yield, but Monero C/C++ code results match that from \emph{Crypto Note Test Address}. Let us see the components and the calculations that take place in order to construct a stealth address.

The example that is presented here was posted by user \verb|skaht| on Monero stack exchange forum~\cite{stackexchange}. The calculations were checked and confirmed.
\begin{itemize}
  \item 256-bit hexadecimal-encoded seed is assumed to be:
  \begin{verbatim}
    198584347013dd91832be3d82529437db7cc8e1850e559cdd3872b29
    ca819601
  \end{verbatim}
  \item Electrum mnemonic words\footnote{These are the words that a wallet owner should remember in order to restore his wallet, if he forgets his password.} corresponding to seed (\verb|./bytes_to_words <above seed>|):
  \begin{verbatim}
    $ ./bytes_to_words 198584347013dd91832be3d82529437db7cc8
    e1850e559cdd3872b29ca819601
  \end{verbatim}
  \begin{tcolorbox}[colback=green!5!white,colframe=green!65!black,title=Output:]
    wallets drinks insult popular fall textbook scoop apology unsafe fifteen cuffs pimple roster nerves pixels upstairs academy sprig eclipse leopard peeled faxed gutter happens roster
  \end{tcolorbox}
  \item Private spend key calculation (\verb|./sc_reduce32 <private spend key>|):
  \begin{verbatim}
    $ ./sc_reduce32 198584347013dd91832be3d82529437db7cc8e185
    0e559cdd3872b29ca819601
  \end{verbatim}
  \begin{tcolorbox}[colback=green!5!white,colframe=green!65!black,title=Output:]
    \small{198584347013dd91832be3d82529437db7cc8e1850e559cdd3872b29ca819601}
  \end{tcolorbox}
  \item Private view key calculation (\verb|./keccak|\footnote{Hash function.}\verb=<private spend key> | ./sc_reduce=):
  \begin{verbatim}
    $ ./keccak 198584347013dd91832be3d82529437db7cc8e1850e559c
    dd3872b29ca819601
    $ ./sc_reduce32 <the keccak output>
  \end{verbatim}
  \begin{tcolorbox}[colback=green!5!white,colframe=green!65!black,title=Output:]
    \footnotesize{889DA12A88D36BCE0966AB1A79125779DD1F2FC6F1145DE131FD52A5B468796D}
    \tcblower
    \small{faa5defce980fdbd03b9dd4841371dfcdc1f2fc6f1145de131fd52a5b468790d}
  \end{tcolorbox}
  \item Public spend key calculation (\verb|./secret_key_to_public_key <private spend key>|):
  \begin{verbatim}
    $ ./secret_key_to_public_key 198584347013dd91832be3d82529437
    db7cc8e1850e559cdd3872b29ca819601
  \end{verbatim}
  \begin{tcolorbox}[colback=green!5!white,colframe=green!65!black,title=Output:]
    \small{b66991d7d7c68513533d0560f820d75adfb0911487ba62274b759f7b3ccd4a90}
  \end{tcolorbox}
  \item Public view key calculation (\verb|./secret_key_to_public_key <private view key>|):
  \begin{verbatim}
    $ ./secret_key_to_public_key faa5defce980fdbd03b9dd4841371
    dfcdc1f2fc6f1145de131fd52a5b468790d
  \end{verbatim}
  \begin{tcolorbox}[colback=green!5!white,colframe=green!65!black,title=Output:]
    \small{3c450f27cd6849d9130addb2c566d910c5ef9bf4cecaed547004496fda52a4ff}
  \end{tcolorbox}
\end{itemize}

Now, we should note that the calculation of the stealth address (hexadecimal format) is:
\begin{verbatim}
  prefix + public_spend_key + view_public_key + keccak_checksum_postfix
\end{verbatim}
The \verb|prefix| in Monero addresses is always $12$ and is a marking of a Monero address. In this context the character \verb|+| is used to mark concatenation of strings. The \verb|keccak_checksum_postfix| computation is:
\begin{itemize}
  \item Stealth address checksum calculation (\verb|./keccak <almost an address>|):
  \begin{verbatim}
    $ ./keccak 12 +
    b66991d7d7c68513533d0560f820d75adfb0911487ba62274b759f7b3ccd4a90 +
    3c450f27cd6849d9130addb2c566d910c5ef9bf4cecaed547004496fda52a4ff
  \end{verbatim}
  \begin{tcolorbox}[colback=green!5!white,colframe=green!65!black,title=Output:]
    \footnotesize{ADD568169DBF2C6D3F595EE8610A189955BECD1EDF150627CBF2F2C49B0AEA71}
  \end{tcolorbox}
  \item So, the hexadecimal format of a Monero stealth address is:
  \begin{verbatim}
    12b66991d7d7c68513533d0560f820d75adfb0911487ba62274b759f7b3ccd4
    a903c450f27cd6849d9130addb2c566d910c5ef9bf4cecaed547004496fda52
    a4ffADD56816
  \end{verbatim}
\end{itemize}

In order to convert a hexadecimal representation of a stealth address in base58 format we calculate the base58 format of each 8 bytes and concatenate the results (presented between brackets). For the conversion, we used \verb|bx| (bitcoin explorer~\cite{bx}):

\begin{enumerate}
  \item \verb|$ bx base58-encode 12b66991d7d7c685| ($\rightarrow$ \verb|48Y3H2eSZ6C|)
  \item \verb|$ bx base58-encode 13533d0560f820d7| ($\rightarrow$ \verb|4EUjY1B5viS|)
  \item \verb|$ bx base58-encode 5adfb0911487ba62| ($\rightarrow$ \verb|GCbCLPcmMiy|)
  \item \verb|$ bx base58-encode 274b759f7b3ccd4a| ($\rightarrow$ \verb|7aD69yqUsaH|)
  \item \verb|$ bx base58-encode 903c450f27cd6849| ($\rightarrow$ \verb|R8GLE3rvSwr|)
  \item \verb|$ bx base58-encode d9130addb2c566d9| ($\rightarrow$ \verb|dJtpZYG1peC|)
  \item \verb|$ bx base58-encode 10c5ef9bf4cecaed| ($\rightarrow$ \verb|3oipCqfUvCc|)
  \item \verb|$ bx base58-encode 547004496fda52a4| ($\rightarrow$ \verb|F89i86kuEjV|)
  \item \verb|$ bx base58-encode ffADD56816| ($\rightarrow$ \verb|Vr5GCdj|)
\end{enumerate}

Finally, we get the Monero stealth address:
\begin{tcolorbox}[colback=green!5!white,colframe=green!65!black]
  \small{48Y3H2eSZ6C4EUjY1B5viSGCbCLPcmMiy7aD69yqUsaHR8GLE3rvSwrdJtpZYG\\
  1peC3oipCqfUvCcF89i86kuEjVVr5GCdj}
\end{tcolorbox}

\section{CryptoNote Transaction} \label{sec:cryptonote_transaction}
%
Here we will present a complete CryptoNote transaction from Bob to Carol. Again reproduced from CryptoNote~\cite{cryptonote}, we will subjoin an example and a figure showing the details. The example below is illustrated in \hyperref[fig:transaction]{figure}~\ref{fig:transaction}, in the next page.

Bob decides to spend an output, which was sent to the one-time public key. He needs Extra \textbf{(1)}, TxOutNumber \textbf{(2)}, and his Account private key \textbf{(3)} to recover his one-time private key \textbf{(4)}.

When sending a transaction to Carol, Bob generates its Extra value by random \textbf{(5)}. He uses Extra \textbf{(6)}, TxOutNumber \textbf{(7)} and Carol's Account public key \textbf{(8)} to get her Output public key \textbf{(9)}.

In the input Bob hides the link to his output among the foreign keys \textbf{(10)}. To prevent double-spending he also packs the Key image, derived from his One-time private key \textbf{(11)}.

Finally, Bob signs the transaction, using his One-time private key \textbf{(12)}, all the public keys \textbf{(13)} and Key Image \textbf{(14)}. He appends the resulting Ring Signature to the end of the transaction \textbf{(15)}.
\clearpage
\pagebreak

\begin{figure}[ht]
  \centering
  \includegraphics[scale=0.31, angle=90, keepaspectratio]{Images/CryptoNote/transaction.png}
  \caption{A sample transaction from Bob to Carol.~\cite{cryptonote}}
  \label{fig:transaction}
\end{figure}
