%
%% Abstract
%
\restoregeometry

\pagenumbering{gobble}
\thispagestyle{empty}
\chapter*{Abstract}
\hyperref[sec:Bitcoin]{Bitcoin} has been a successful implementation of the concept of peer-to-peer electronic cash. Based on this technology several cryptocurrency projects have arisen, each one focusing on its purposes and goals. \hyperref[sec:Monero]{Monero} is a decentralized cryptocurrency focusing on privacy and anonymity.

In a world of surveillance, Monero raises the alarm about one of the fundamental human rights, which is continuously violated, privacy. In addition, Monero is built to achieve equality between \hyperref[sec:mining]{miners}. Corporations are taking over almost every successful cryptocurrency, by making mining participation harder and harder for the hobbyists and supporters. Monero tries to keep its community clean of unhealthy competition. This is achieved through \hyperref[sec:egalitarian]{egalitarianism}, which is based οn a cryptographic mining function.

This function is called \hyperref[ch:cryptonight]{CryptoNight} and is part of the \hyperref[sec:CryptoNote]{CryptoNote} protocol, the heart of Monero's structure. The feature of this function that makes it egalitarian is a cryptographic property, named \hyperref[sec:memory-hard]{memory-hardness}. CryptoNight is alleged to be memory-hard. But, still today, this is just a claim.

We put to the test this claim, trying to construct a formal mathematical proof, but we fail to do so. We discuss the reasons for our failure and try to use them to construct an attack on this feature. To our knowledge, we are the first to study this CryptoNight's property and the first to present graphically all the stages of CryptoNight's functionality.

Finally, we present the knowledge gained and wish for this document to be useful in the future to colleagues that want to contribute in this field. The aim of this work is to contribute to Monero's fight for privacy, anonymity and equality.
\clearpage
%
%% Empty page
%
\thispagestyle{empty}
\null
\clearpage
