\setcounter{chapter}{-1}
\chapter{Preface}
%
\section{Why Monero?}
I would like to take a moment here and share the experience of selecting a research topic. This was as important to me as the rest of my work. I am not talking about concepts of "good topic" or a list of "factors to consider". I am definitely not an expert on this subject and you can find many valuable information about this process online. I would like to share with you the impact of an argument, about selecting this topic over other options, that was addressed to me by my colleague and friend, Dionysis Zindros. He said to me that in his opinion this topic would be "beneficial for the Monero community".

I strongly recommend before you select your topic of research to take an evening of your time and read a paper titled "The Moral Character of Cryptographic Work", written by Phillip Rogaway~\cite{moral}. No matter how small this world makes you feel, when you select a topic you have responsibility. I was lucky to work with people who understood this and led me to take a moment and think about what cause I really wanted to contribute to.

In our days, cryptographic work is a political action. There is no doubt about that. Think through your intentions and the person you want to be before you select your path. Remember, you have responsibility.
%
\section{An important thank you note}
During my research I had a lot of help from forum answers and conversations and most of all, from the monero stack exchange users~\cite{stackexchange}. I would like to thank many users for their valuable share of knowledge. I had help from a lot of users and several stack exchange forums, who pointed me to the right direction or helped me clarify my misunderstanding of notions from time to time.
\pagebreak

I want to thank them for the aid, but most important, I want to thank everyone who contributes to this knowledge sharing. It means a lot to me and gives me a perspective on my occupation. I don't see the work of a cryptographer as an 8-hour employment to make ends meet. I hope my interests and efforts to be much more than a day job. I feel part of a community that contributes to the real world and keeps me motivated and convinced that our work makes the world a better place to live in. I thank you all for your example.
%
\section{Narrative}
I will change the first person singular narrative. Apart from all the people I have already thanked, my personality has been shaped by my family, my friends, important acquaintances and stories about my heros. All of them are part of me, part of who I am.

As a result, I see this work as a collaboration of all of us and that completes the list of the reasons why I will keep a first person plural narrative from now on. I could not do that in the above paragraphs because this section, as you can understand, is extremely personal. I thank you all.

Another change will be the way I am addressing to you, my reader. It will be, from now on, in a third person point of view. This is just a personal aesthetic choice. I thank you for taking the time to read our work. I hope you find it enjoyable.
