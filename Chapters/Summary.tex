\chapter{Summary}
%
\epigraph{Arguing that you don't care about the right to privacy because you have nothing to hide is no different than saying you don't care about free speech because you have nothing to say.}{\textit{Edward Snowden}}
%
\section{Knowledge gained}
%\setlength{\intextsep}{0pt}
\begin{wrapfigure}{L}{0.26\textwidth}
\centering
\includegraphics[width=0.26\textwidth]{Images/Summary/knowledge.jpg}
\end{wrapfigure}
The intuition behind the nature of the problem lies in the relation between the first and the second address written in each loop, in the second stage of the function. As we saw, the first address is a random pointer produced from the input. This is something that can be leveraged to some adversary's advantage, at least in the first round. But the way the second pointer is produced makes it hard to keep track of the computation.

Now, let's take a step back. If we want to attack the memory-hardness property, what exactly is our goal? Well, the train of thought is the following: If we managed to compute the first 128 sequential bytes of the state of the scratchpad (after the second stage), without using the rest of it, then it would be a win. We can compute in steps the digest of the CryptoNight function, using just 128 bytes of the scratchpad in each step.

We can calculate the initialization of the first 128 bytes of the scratchpad, then move on to the second stage and finally compute the first part of our solution, running just the first step of the third stage. Then, we repeat the above for every other 128 bytes of the scratchpad (15,625 times). This will give us the digest of the function, using 256 bytes of memory (2$\cdot$128, in order to remember last round's outcome) and time complexity of the same order of magnitude as the proposed implementation. That would complete a successful attack.
\pagebreak

The reason we failed is that it seems impossible to control the pointers to the addresses in the second stage. We cannot compute the final stage of the scratchpad, 128 bytes at a time. We need all of it to produce a correct computation. Let's see the details.

Without loss of generality, let the first address be $a$ and the second address be $b$, within some loop. Then,
\begin{equation} \nonumber
  b = AES(a,SP_a)
\end{equation}

Under the assumption that AES is a PRF, the above can be leveraged as we described but for one round at most. Then, control is lost. The reason is that the output of a PRF cannot be guessed with more than negligible probability. That means that we can't control the sequence of the addresses written in the second stage. Thus, we can't compute partially the final stage of the scratchpad. It has to be used as a whole.

\section{Future Work}
At first, our intuition was that the next step towards the goals of this thesis is to analyze the relation between the inputs and the outputs in each round. Maybe there is something there, we could not find. But with a closer look, it seems that if we cannot "fix" the relation to the equality one, then any control we might achieve by this analysis is lost in the next round, due to our assumption that AES is a PRF.

It is obvious that a successful attack on the AES function could lead to a successful attack on CryptoNight's memory-hardness property. However, AES being a PRF, is a reasonable assumption and it is expected to be hard to find a vulnerability in AES and a distinction between AES and PRFs. AES is a cryptographic primitive that has been thoroughly reviewed and checked.

We don't know yet, whether the reverse is true. Namely, whether a successful attack on CryptoNight's memory-hardness property is a step towards a successful attack on AES. This is a step that can follow our work, but it seems a really tricky and difficult problem.

\subsubsection{Fresh point of view}
%\setlength{\intextsep}{0pt}
\begin{wrapfigure}{L}{0.28\textwidth}
\centering
\includegraphics[width=0.28\textwidth]{Images/Summary/nothing.jpg}
\end{wrapfigure}
If someone wants to continue our research, we recommend a huge step back. Maybe another model for the problem or something that is not produced from a similar point of view. As we presented in this thesis, we have found problems very early in our analysis and we couldn't pass beyond the second round of the second stage.

If someone wants to perform a cryptographic analysis on CryptoNight's memory-hardness property, then he/she is supposed to analyze this second stage. This second stage is the place where the origins and the basis of the memory-hardness property lie.
\clearpage
\pagebreak

\section{What's new?}
This thesis is the first, to our knowledge, to analyze the memory-hardness property of the CryptoNight function. We hope that our analysis is helpful to the researcher or the researchers, who would like to continue this effort. We tried to make this document a good start for further exploration of this interesting and valuable problem.

We wish good luck to the people that will continue this work and help the Monero community in their effort for a better world. Surveillance amplification in the physical and digital world is apparent in our days and we believe that progress in research for anonymity and privacy in the public domain is something that many people seek. After all, privacy is a fundamental human right that many - if not all - human rights are based on. Without privacy, human rights, i.e., \emph{freedom of speech}, collapse.
