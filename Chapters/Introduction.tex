\chapter{Introduction}
%
\epigraph{If you don’t believe it or don’t get it, I don’t have the time to try to convince you, sorry.}{\textit{Satoshi Nakamoto}}
%

\section{Decentralization}

\section{Our contribution}

\section{Thesis structure}
We tried to make this document readable. From this point of view, we tried to achieve completeness, in the sense that the reader should not refer to external resources to understand the basic arguments involved. Of course, a more keen reader can find references for a detailed study of the notions involved in our work, but that is something that we wanted to be as optional as possible.

In the first chapter, the reader will find an overview of the first complete implementation of a cryptocurrency. That was a paper published under the pseudonym Satoshi Nakamoto~\cite{Nakamoto_bitcoin:a}, introducing the \hyperref[sec:Bitcoin]{Bitcoin} project to the world. This overview is certainly not a detailed description of all aspects of Bitcoin, as this would be unproductive for the discussion in this work. However, we believe that we give a satisfactory description of the project and we hope that the reader will acquire an adequate understanding of the structure of this novelty.

In the second chapter, we introduce the reader to the concept of \hyperref[sec:mining]{mining}, based on the knowledge gained through the study of the first chapter. The purpose of this description is to introduce the notion of \emph{egalitarianism} and more precisely, the notion of \hyperref[sec:egalitarian]{egalitarian mining}. This notion was for a long time a folklore subject of dispute in the community, being a starter for many passionate conversations. It was formally defined recently, in the work of Dimitris Karakostas, Aggelos Kiayias, Christos Nasikas and Dionysis Zindros ~\cite{egalitarianism}.

In the third chapter we introduce another cryptocurrency, named \hyperref[sec:Monero]{Monero}, and the purposes of its community along with a technical description of its features. The basic structure for Monero project is the \hyperref[sec:CryptoNote]{CryptoNote} protocol, described in a paper published under the pseudonym Nicolas van Saberhagen~\cite{citeulike:14139412}. We describe its features as well.

Then, we get to the point where we can discuss the problem of interest of this thesis. This is Monero's mining function, \hyperref[ch:cryptonight]{CryptoNight}. This function is part of the CryptoNote protocol and it is the reason that Monero claims to offer egalitarianism in its mining process. This feature's existence is due to a known property, called \hyperref[sec:memory-hard]{memory-hardness}. This function is alleged to be memory-hard, although we found no formal research on this matter. CryptoNight's functionality and features are described in chapter four.

In the next chapter, we try to construct a formal mathematical proof of the CryptoNight's memory-hardness property and we discuss the reasons we failed to do so. Using these reasons, we attempt to construct an attack on this property.

Finally, we sum up our observations and try to highlight important knowledge gained through this journey. In the first appendix, the reader can find formal mathematical definitions of the notions used or referred, in an effort for this thesis to be both readable and complete. Constructions and details about the Monero project, can be found in the second Appendix. However, this information is not needed for the reader to keep up with the flow of our work or to understand our remarks. Nevertheless, they were included for the sake of completeness and for the reader, as he/she might find this information beneficial.
