\setcounter{chapter}{-1}
\chapter{Introduction}
%
\section{Overview}
%
%
%
\section{Thesis structure}
We tried to make this document readable. From this point of view, we tried to achieve completeness, in the sense that the reader should not refer to external resources to understand the basic arguments involved. Of course, a more experienced reader can find refererances for further understanding the notions involved in our work, but that is something that we wanted to be as optional as possible.

In the first chapter, the reader will find an overview of the first complete description of a cryptocurrency. That was Satoshi Nakamoto's paper~\cite{Nakamoto_bitcoin:a}, introducing to the world the Bitcoin project. It is certainly not a detailed description of all aspects of Bitcoin, as this would be unproductive for the problem discussed in this work. However, we believe that we give a satisfactory overview about the Bitcoin project and we hope that the reader will aquire an adequate understanding of the structure of this construction.

In the second chapter, we introduce the reader to the concept of mining, based on the knowledge gained through the read of the first chapter. The purpose of this description is to introduce the notion of egalitarianism and more precisely, the notion of egalitarian mining. This notion was for a long time a folklore subject of dispute in the community, being a starter for many passionate conversations. It was formally defined recently, in the work of Dimitris Karakostas, Aggelos Kiayias, Christos Nasikas and Dionysis Zindros ~\cite{egalitarianism}.

In the third chapter we introduce another cryptocurrency, named Monero, and the purposes of its community along with a technical description of its features. The basic structure for Monero project is described in CryptoNote protocol. We describe its features as well.

Then, we get to the point where we can discuss the problem of interest of this thesis. This is Monero's mining function, CryptoNight. This function is part of the CryptoNote protocol and it is the reason that Monero claims to offer egalitarianism in its mining process. This feature's existence is due to a known property, called memory-hardness. This function is alleged to be memory-hard, although we found no formal research on this matter. CryptoNight's functionality and features are described in chapter four.

In the next chapter, we try to construct a formal mathematical proof of the CryptoNight's memory-hardness property and we discuss the reasons we fail. Using these reasons for the aforementioned failure, we attempt to construct an attack on this property.

Finally, we sum up our observations and try to highlight important knowledge gained through this journey. In the first appendix, the reader can find formal mathematical definitions of the notions used or refered, which might be needed for this thesis to be readable. Constructions and details about the Monero project that are interesting, can be found in the second Appendix. However, this information is not needed to keep up with the flow of our work or to understand our remarks.
