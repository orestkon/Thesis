%%%%%%%%%%%%%%%%%%%%%%%%%%%%%%%%%%%%%%%%%%%%%%%%%%%%%%
% For this chapter I need to build the figures.
% NOT done yet.
% Also need to be checked, but this will happen after
% the figure insertion.
%%%%%%%%%%%%%%%%%%%%%%%%%%%%%%%%%%%%%%%%%%%%%%%%%%%%%%

\chapter{Cryptonight} \label{ch:cryptonight}
%
\epigraph{Democracy must be something more than two wolves and a sheep voting on what to have for dinner.}{\textit{James Bovard}}
%
\section{Description}
In this section we will describe in detail the proposed implementation of the CryptoNight hash function. This function is used in the Monero project in order to achieve \hyperref[sec:egalitarian]{\emph{egalitarian mining}}. It is easy to understand why we characterize this implementation as \emph{proposed}, since each miner is free to use any implementation he/she can think of, as long as it produces the right result.

In order to prove that a function is \hyperref[sec:memory-hard]{\emph{memory-hard}} (see \hyperref[sec:memory-hard]{section}~\ref{sec:memory-hard}) we need to show that no implementation exists, that can produce the same result using less memory without a significant time cost. In other words, using an implementation which needs less memory is not something that can give advantage to some miner because the time factor will make the procedure equally or more "expensive", even if the miner uses parallel computation techniques. Reproduced from CryptoNote~\cite{cryptonight}:
\begin{verbatim}
  CryptoNight is a memory-hard hash function. It is designed to be
  inefficiently computable on GPU, FPGA and ASIC architectures.
\end{verbatim}

In the proposed implementation, a scratchpad\footnote{A large area of memory used to store intermediate values during the evaluation of a memory-hard function.} is used (2MB) to ensure that the memory needed fits the size of L3 cache (per core) in modern processors. In practice, the miner should measure mining power and calculate efficiency.

In Monero mining, CPUs cores are only efficient if they can use the super fast 2MB cache over and over. Each core needs about 2MB for CryptoNight to stay cached. So a miner should check how much L2 cache or in rare cases also L3 cache the CPU has. Then divide by 2MB and this will be how many cores he/she can run at the same time.

There are several reasons to suspect that CryptoNight could be a memory-hard function. One of the most popular argument was that a megabyte of internal memory is an almost unacceptable size for a modern ASIC pipeline. But, hardware is evolving and eventually there was recently an effort for Monero ASIC production. We could not find documents of the first construction of ASIC for Monero, but there was a time that many miners used ASICs and that was easy to see. Observing the raise of hashrate in the Network, it was obvious that there were ASICs used for mining.

There were some thoughts like "\emph{How did they did this? Isn't CryptoNight memory-bound?}". Well, one thing is that CryptoNight is \emph{ASIC-resistant}, not \emph{ASIC-proof}. But, that was not the problem. Another thought is that L3 cache supports a lot of extra functionality like being shared across cores, writing back to RAM, being behind two other levels of cache, etc. which all makes it a lot less efficient (among other issues with the approach). But, again, that was not the case in that particular effort.

L3 latency wasn't the issue. ASICs just traded latency for bandwidth the same way GPUs do. They were built on stacks of DRAM, not lightning fast caches. The costs of cache complexity aren't only latency but also power usage and die space. Raw speed isn't even necessarily the goal for either CPUs or GPUs or ASICs here, it is efficiency.

But, Monero project reacted and announced upgrades bi-annually in order to keep ASICs at bay. Upgrades are a problem, because upgrades produce bugs and vulnerabilities. Especially, when they are that frequent. On the other hand, upgrades in Monero are minor with no changes in the memory-hard part. From this experience, we understand that a formal proof or even a better understanding of the memory-hardness property in practice for its mining function is vital, in order to protect a cryptocurrency from centralization.

In this chapter we will just show the proposed implementation of CryptoNight without any analysis. We will demonstrate the three stages of the computation and the role for each element. A really quick overview of these stages would be something like this:

\begin{enumerate}
  \item Initialize the scratchpad in a pseudo-random manner.
  \item Read/write operations at pseudo-random addresses. (\hyperref[sec:memory-hard]{\emph{memory-hard}} part)
  \item Use all the computations' results to produce the output.
\end{enumerate}

\section{The three stages}
Enough with the overview of the function and its history! Let's dive into it and see in detail its components as it is described in \cite{cryptonight}.

The input of this algorithm is a block and if the value of the Cryptonight function satisfies the target (see \hyperref[eq:target]{equation}~\ref{eq:target}), it is possible that this block is the next block in the blockchain.

\begin{equation}
  \label{eq:target}
  \color{Bittersweet} \mbox{Cryptonight}
  \color{black} (
  \color{RedViolet} \mbox{block}
  \color{black} )\leq
  \color{ForestGreen} \mbox{Target}
  \color{black}
\end{equation}
So, the input of the function is a block of transactions along with the necessary fields, which are specified by the Monero protocol. For our purposes, it is enough for the reader to think of the procedure as simple as it gets. We accept that the only way to meet the target is by bruteforcing. So, the miner tries many blocks as "candidates" and hopes for the best. Every time he/she "tries", he/she actually computes the Cryptonight digest for some random block and checks whether the \hyperref[eq:target]{equation}~\ref{eq:target} is satisfied.

\subsection{The first stage}
The first stage of the algorithm sets the initial value of the scratchpad. In order to prevent several attack schemes, the scratchpad must be initialized with data chosen in a way, which is indistiguishable from the uniform distribution. This is the goal.

We will describe the first stage in several parts and discuss the role of each part and its contribution regarding the properties of function's output. To begin, let's prepare the tools:
\begin{enumerate}
  \item \label{hashing} Hash the input using Keccak~\cite{keccak} ($b=1600$, $c=512$).
  \item Choose the first 32 bytes of the final state.
  \item Interpret them as an AES-256 key.
  \item Expand them to 10 round keys.
\end{enumerate}

Keccak is the versatile cryptographic function that is most known as SHA-3. The parameter analysis and the description of their part is beyond the scope of this thesis. The reader is refered to their work.

We will consider Keccak a collision-free hash function. The next three steps produce random keys for encryption. We consider these keys random enough in the way that they are interpreted as keys and expanded according to \cite{nla.cat-vn4183631}. Create the scratchpad:
\begin{enumerate}
  \setcounter{enumi}{4}
  \item Allocate 2097152 bytes (2MiB).
\end{enumerate}
The encryption part:
\begin{enumerate}
  \setcounter{enumi}{5}
  \item Split the bytes 64 to 191 into 8 blocks of 16 bytes each.
  \item \label{step 7} Encrypt the blocks as follows:
    \begin{verbatim}
      for i = 0..9 do:
          block = aes_round(block, round_keys[i])
    \end{verbatim}
\end{enumerate}
\begin{enumerate}
  \setcounter{enumi}{7}
  \item Fill 128 bytes of the scratchpad with the resulting blocks.
\end{enumerate}
Repeat:
\begin{enumerate}
  \setcounter{enumi}{8}
  \item With the resulting blocks run \hyperref[step 7]{step}~\ref{step 7} again.
\end{enumerate}

Each time 128 bytes are written, they represent the
result of the encryption of the previously written 128 bytes. The
process is repeated until the scratchpad is fully initialized.

\subsection{The second stage (memory-hardness)}
The second stage of the algorithm uses the initialized scratchpad and two values that are computed from the hashed input of the function. Its goal is to perform computations on the scratchpad values (on all of them with high probability) and produce a final scratchpad structure that can't be computed otherwise or in stages (without huge time complexity). The memory-hardness property is satisfied if and only if there is no other way to compute the final state of the scratchpad using memory less than the size of the scratchpad. That is the intuition. Let's see the details.

\paragraph{(The preparation part)} The core structure of this stage is a loop. However, before illustrating the computations that take place inside the loop, there are some computations needed for preparation and two technical clarifications.

\begin{enumerate}
  \item \label{memh: step 1} Compute the values of $a$ and $b$.
\end{enumerate}
Elements $a$ and $b$ are the two values which, along with the scratchpad, are given as input to the loop. More specifically, the first 64 bytes of the hashed input (the Keccak state) are split in two parts (32 bytes each part) and XOR-ed ($\oplus$), and the resulting 32 bytes are used to initialize variables $a$ and $b$, 16 bytes each.

\noindent \emph{(Clarification 1).} The reader may notice in figure ... that the function uses a 16-byte value as an address in the scratchpad. Actually, the value is interpreted as a little-endian integer. The 21 low-order bits are used as a byte index. To ensure the 16-byte alignment, the four low-order bits of the index are cleared. This alignment is essential, as the data is read from and written to the scratchpad in 16-byte blocks.

\noindent \emph{(Clarification 2).} The main loop is iterated $524,288 = 2^{19}$ times. Every time, two blocks of the scratchpad are written, so with high probability, the whole scratchpad will be overwritten. In every iteration, along with the two blocks of the scratchpad, values $a'$ and $b'$ are computed, which are used as input to the next iteration.

Now we are ready to describe the inner computations of the loop. We will divide this stage into parts, as it will help us later in the analysis. The number of parts are determined, based on some intermediate values. Here, we note that the intermediate values are not memory requirements, as we can implement the computations with only 32 bytes of memory (for $a$ and $b$) plus the memory needed for the scratchpad. But during theoretical analysis and understanding of the function's computations, these intermediate values seem natural stops of the train of thought. So, after the \hyperref[memh: step 1]{step}~\ref{memh: step 1}:

\begin{enumerate}
  \setcounter{enumi}{1}
  \item \label{memh: step 2} Interpret the value of $a$ as a scratchpad address.
  \item \label{memh: step 3} Read from this address.
  \item Evaluate the AES function with data from \hyperref[memh: step 3]{step}~\ref{memh: step 3} and key the value of $a$.
\end{enumerate}
Let's call this intermediate value $c$. And let's add a final step to this part:
\begin{enumerate}
  \setcounter{enumi}{4}
  \item Calculate $c \oplus b$ and write the result to the address of \hyperref[memh: step 2]{step}~\ref{memh: step 2}.
\end{enumerate}
The value $c$ is passed as $b'$, part of the input of the next iteration. The second part involves another read from the scratchpad:

\begin{enumerate}
  \setcounter{enumi}{5}
  \item \label{memh: step 6} Interpret the value of $c$ as a scratchpad address.
  \item Read from this address. (We will refer to this intermediate value as $d$.)
  \item \label{memh: step 8} Multiply\footnote{The multiplication uses only the first 8 bytes of each argument, which are interpreted as unsigned 64-bit little-endian integers and multiplied together. The result is converted into 16 bytes, and finally the two 8-byte halves of the result are swapped~\cite{cryptonight}.} $c$, $d$ and add the value of $a$ to the result.
  \item Write the result of \hyperref[memh: step 8]{step}~\ref{memh: step 8} to the address of \hyperref[memh: step 6]{step}~\ref{memh: step 6}.
\end{enumerate}
This concludes the second part. The scratchpad is written twice per iteration. The only thing that is left to conclude the description of the second stage, is the computation of $a'$, part of the input of the next iteration. This is computed as follows:

\begin{enumerate}
  \setcounter{enumi}{9}
  \item Compute $d \oplus \mbox{(<the result of \hyperref[memh: step 8]{step}~\ref{memh: step 8}>) }$ to compute $a'$.
\end{enumerate}

\subsection{The third stage}
The third stage of the algorithm uses the final state of the scratchpad to produce the output. During this stage AES operation is used. At first, the function extracts 10 key values from 32 bytes of the hashed input of the function, similar to the \hyperref[hashing]{step}~\ref{hashing} of the first stage.

\noindent Extracting keys:
% Extract keys
\begin{enumerate}
  \item Choose the bytes $[32...63]$ of the final Keccak state.
  \item Interpret them as an AES-256 key.
  \item \label{keys} Expand them to $10$ round keys.
\end{enumerate}
After this, the function needs a starting value. It applies XOR-operation($\oplus$) on bytes $[64...191]$ of the hashed input and the first 128 bytes of the scratchpad. Let's call these values \emph{input} and \emph{scratchpad[0]}.

\begin{enumerate}
  \setcounter{enumi}{3}
  \item \label{xor} \emph{input} $\oplus$ \emph{scratchpad[0]}.
\end{enumerate}
Now, using the first key of \hyperref[keys]{step}~\ref{keys} as key to the AES operation:

\begin{enumerate}
  \setcounter{enumi}{4}
  \item \label{encrypt} Encrypt the result of \hyperref[xor]{step}~\ref{xor}.
\end{enumerate}
Repeat the last two steps as follows:

\begin{itemize}
  \item Take as input the encrypted result of the last step.
  \item Take the next 128 bytes of the scratchpad (\emph{scratchpad[1]}).
  \item Use, as AES key, the next extracted key.
  \item Execute \hyperref[xor]{step}~\ref{xor} and \hyperref[encrypt]{step}~\ref{encrypt}.
\end{itemize}
until the last bytes of the scratchpad are used to the aforementioned operations. After the last bytes of the scratchpad are XOR-ed and encrypted,

\begin{enumerate}
  \setcounter{enumi}{5}
  \item \label{modified} Use the result to replace the bytes $[64...191]$ of the hashed input.
\end{enumerate}
We we call the state of the hashed input after the above step as the \emph{modified Keccak state}. To produce the final result the function performs the next steps.

\begin{enumerate}
  \setcounter{enumi}{5}
  \item Pass the \emph{modified Keccak state} through Keccak-$f$ (the Keccak permutation~\cite{keccak}).
  \item Choose the 2 low-order bits of the first byte of the \emph{modified Keccak state}.
  \item Based on these bits choose a hash function:
  \begin{itemize}
    \item case 00: BLAKE-256
    \item case 01: Groestl-256
    \item case 10: JH-256
    \item case 11: Skein-256
  \end{itemize}
  \item \label{output} Apply the chosen function to the \emph{modified Keccak state}.
\end{enumerate}
The result of \hyperref[output]{step}~\ref{output} is the output of the CryptoNight function. For more information about these functions, the reader is refered to the respective articles~\cite{10030667226,sha3groestl,sha3W09,sha3F+08}.

\section{Analysis}
Now that we have described the way this function computes its output, we will try to build a model for our theoretical analysis. There are several assumptions that we have to make to abstract the function's operation. For this purpose, we build our model and define its security. At first, we have to focus on the part that is linked to the \hyperref[sec:memory-hard]{\emph{memory-hardness}} property (see \hyperref[sec:memory-hard]{section}~\ref{sec:memory-hard}). The reader now should have an understanding about the general purpose of each stage. The first stage sets the scene for the memory-hardness part. We will assume that it initializes the scratchpad in a way, which is indistiguishable from the uniform distribution. It is also safe to assume that, for an honest miner, the input is chosen uniformly at random also. Our goal is to focus on the second stage, analyze it and then try to imagine, how an adversary miner can attack the memory-hardness property of this function. From the aforementioned assumptions we can conclude that the input of the second stage is chosen uniformly at random from its domain. Just to remember, the second stage's input is:
\begin{itemize}
  \item $a$
  \item $b$
  \item Scratchpad, from now on denoted as $SP$
\end{itemize}

\subsection{Parameters}
One of the first things that we need to do is to parametrize the input. We can't talk about complexity or security without the relative size between our objects or calculations. Moreover, this kind of analysis can help to generalize results and conclusions.

We will arbitrarily choose $n$ as symbol for the size of $a$ and fix everything else respectively. With that analysis in mind, we are fixing our language. Symbols:

\[
  \begin{array}{lcl}
    \mbox{Size of }a \mbox{: } & \: & n \\
    \mbox{Size of }b \mbox{: } & \: & n \\
    \mbox{Size of }SP \mbox{: } & \: & \beta_{n} = 2 \cdot 2^{20} = 2 \cdot n^{5} \mbox{ (polynomial)} \\
    \mbox{Value of }SP \mbox{ in address }x \mbox{: } & \: & SP_{x} \mbox{ (of size }n\mbox{)} \\
  \end{array}
\]

\subsection{AES as PRF}
The Advanced Encryption Standard (AES), also known by its original name Rijndael~\cite{Daemen99aesproposal:}, is a specification for the encryption of electronic data established by the U.S. National Institute of Standards and Technology (NIST) in 2001~\cite{nla.cat-vn4183631}.

AES is a subset of the Rijndael block cipher developed by two Belgian cryptographers, Vincent Rijmen and Joan Daemen, who submitted a proposal to NIST during the AES selection process. Rijndael is a family of ciphers with different key and block sizes.

For AES, NIST selected three members of the Rijndael family, each with a block size of 128 bits, but three different key lengths: 128, 192 and 256 bits.

AES has been adopted by the U.S. government and is now used worldwide. The algorithm described by AES is a symmetric-key algorithm, meaning the same key is used for both encrypting and decrypting the data.
\pagebreak

In CryptoNight function AES is used for its properties as a cryptographic function. If a different key and a different input is chosen every time, then it is safe to assume that no prediction about the output of AES on this input can exist. To make this assumption more formal, we assume that AES is a \hyperref[sec:PRF]{pseudorandom function} (\hyperref[sec:PRF]{PRF}, see \hyperref[sec:PRF]{section}~\ref{sec:PRF}) and the seed is the key of AES.

%%%%%%%%%%%%%%%%%%%%%%%%%%%%%%%%%%%%%%%%%%%%%%%%%%%%%
% Here you must declare the game and the Security
% model for this assumption. Check the photos on
% the Desktop's desktop.
%%%%%%%%%%%%%%%%%%%%%%%%%%%%%%%%%%%%%%%%%%%%%%%%%%%%%

\subsection{Operations}
Apart from the AES use, the memory-hard stage of CryptoNight function performs one addition, two XOR-operations and one multiplication. In order to be able to produce conclusions, we try to analyze what are the side effects of these operations. All of these operations get two inputs of size 16 bytes and produce a 16 byte result.

Let's examine them one by one. In the case of the XOR operation, it is easy to see that if the two inputs are chosen uniformly at random then the result is also uniformly at random chosen. In the case of addition, reproduced from CryptoNote~\cite{cryptonight}:
\begin{verbatim}
   The 8byte_add function represents each of the arguments as a
   pair of 64-bit little-endian values and adds them together,
   component-wise, modulo 2^64. The result is converted back into
   16 bytes.
\end{verbatim}

It is trickier to see the same here, but with a little effort one can see that if the input is chosen uniformly at random then the result is uniformly at random chosen, too. In the case of multiplication, reproduced from CryptoNote~\cite{cryptonight}, again:
\begin{verbatim}
   The 8byte_mul function, however, uses only the first 8 bytes
   of each argument, which are interpreted as unsigned 64-bit
   little-endian integers and multiplied together. The result is
   converted into 16 bytes, and finally the two 8-byte halves of
   the result are swapped.
\end{verbatim}

The last case, is more complex. At first we notice the following: if one of the inputs is null then the result is also null. That could be a problem. We try to calculate the probability that the result is null, due to the null value of one or both inputs. We don't care about the null value of the result due to modulo operation. That probability is obviously equal to the probability that the result is equal with some other value. We would like the "extra" null probability because of the input's null value, to be negligible.

Due to the special way CryptoNight function performs the multiplication, the inputs' size are 64 bits = 8 bytes = $\frac{n}{2}$. In addition, considering that AES is a PRF (assumption) and $SP$ is unifomly random (at least at the first round by assumption), then the two inputs are independent. The probability of one or both inputs to be null is:
\begin{equation} \label{prob_zero}
  \frac{1}{2^{n/2}} + \frac{1}{2^{n/2}} - \frac{1}{2^{n/2} \cdot 2^{n/2}} = \frac{2\cdot2^{n/2}-1}{2^n} < \frac{1}{2^{(n/2)-1}} = \mbox{\textbf{negl(n)}}
\end{equation}

That seems to be fine for our analysis. But, the next problem is this: Multiplication of two 8 byte numbers produce a result modulo 16 bytes (mod $2^{128}$). The reader can see that the probability of a value to be an output of the multiplication declines, as the values grow. Even after the swap that is performed at the end of the multiplication the problem persists. We have not a uniformly at random distributed result, even if the inputs are chosen uniformly at random.

The above multiplication is perfomed this way because of the modern CPU registers' size. The 8 bytes multiplication is optimized. ASICs couldn't do this as fast as a modern CPU could, but technology advances and now there are chips that do the same computation roughly with the same time cost.

%%%%% Please fix the way this paragraph expresses the notion.
However, after our theoretical analysis and the thought that the time cost of a 16 byte multiplication or maybe an 8 byte multiplication implementation that maps inputs' uniform distribution can make a system less efficient against ASICs but not completely inefficient, we should propose something better. But that is a detail and does not make a great difference in our analysis. So, from now on, we can assume that the above problem is solved and we have a multiplication implementation that produces a uniformly at random distributed result.

% You will explain probabilities in the next chapter, but still, you can analyze the game perhaps?
% Definition of security ...? both AES as PRF and memory-hardness (don't know if
% it is needed here, check preliminaries)
% A glimpse on what set of variables are NOT independent, or NOT DEFINITELY independent. (Maybe this goes to the proof section)
% Just set up your model, so you can write next chapter easily...
%% Still needs figures...
